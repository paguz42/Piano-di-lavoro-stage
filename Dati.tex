%----------------------------------------------------------------------------------------
%   USEFUL COMMANDS
%----------------------------------------------------------------------------------------

\newcommand{\dipartimento}{Dipartimento di Matematica ``Tullio Levi-Civita''}

%----------------------------------------------------------------------------------------
% 	USER DATA
%----------------------------------------------------------------------------------------

% Data di approvazione del piano da parte del tutor interno; nel formato GG Mese AAAA
% compilare inserendo al posto di GG 2 cifre per il giorno, e al posto di 
% AAAA 4 cifre per l'anno
\newcommand{\dataApprovazione}{gg-mm-aaaa}

% Dati dello Studente
\newcommand{\nomeStudente}{Manuel}
\newcommand{\cognomeStudente}{Pagotto}
\newcommand{\matricolaStudente}{1144708}
\newcommand{\emailStudente}{manuel.pagotto@studenti.unipd.it}
\newcommand{\telStudente}{+ 39 393 37 43 936}

% Dati del Tutor Aziendale
\newcommand{\nomeTutorAziendale}{nomeTutor}
\newcommand{\cognomeTutorAziendale}{cognomeTutor}
\newcommand{\emailTutorAziendale}{mail@mail.it}
\newcommand{\telTutorAziendale}{+ 39 000 00 00 000}
\newcommand{\ruoloTutorAziendale}{}

% Dati dell'Azienda
\newcommand{\ragioneSocAzienda}{Azienda S.p.A}
\newcommand{\indirizzoAzienda}{Via Roma 1, Roma (RM)}
\newcommand{\sitoAzienda}{http://example.com/}
\newcommand{\emailAzienda}{mail@mail.it}
\newcommand{\partitaIVAAzienda}{P.IVA 12345678999}

% Dati del Tutor Interno (Docente)
\newcommand{\titoloTutorInterno}{Prof.}
\newcommand{\nomeTutorInterno}{Alessandro}
\newcommand{\cognomeTutorInterno}{Sperduti}

\newcommand{\prospettoSettimanale}{
     % Personalizzare indicando in lista, i vari task settimana per settimana
     % sostituire a XX il totale ore della settimana
    \begin{itemize}
        \item \textbf{I Settimana - Introduzione (24 ore)}
        \begin{itemize}
            \item Incontro con persone coinvolte nel progetto per discutere i requisiti e le richieste relativamente al sistema da sviluppare;
            \item Verifica credenziali e strumenti di lavoro assegnati;
            \item Presa visione dell'infrastruttura esistente;
        \end{itemize}

        \item \textbf{II Settimana - Formazione 1 (24 ore)} 
        \begin{itemize}
            \item Oracle Fundamentals of Database in OCI - Getting Started witch Autonomous Database;
            \item Oracle Fundamentals of Database in OCI - Autonomous Database on Shared Infrastructure;
        \end{itemize}

        \item \textbf{III Settimana - Formazione 2 (24 ore)} 
        \begin{itemize}
            \item Oracle Fundamentals of Database in OCI - Autonomous Database on Dedicated Infrastructure;
            \item Oracle Fundamentals of Database in OCI - Managing and Monitoring Autonomous Database;
        \end{itemize}

        \item \textbf{IV Settimana - Formazione 3 (24 ore)} 
        \begin{itemize}
            \item Oracle Fundamentals of Database in OCI - DB Systems:
            \begin{itemize}
                \item Database Concepts;
                \item Creating a Database System;
                \item Managing DB Systems;
                \item Oracle Data Guard;
                \item Securing the Database Systems;
            \end{itemize} 
            \item Oracle Fundamentals of Database in OCI - MySQL;
        \end{itemize}

        \item \textbf{V Settimana - Formazione 4 (24 ore)} 
        \begin{itemize}
            \item Oracle Fundamentals of Database in OCI - NoSQL and others;
            \item Oracle Cloud Database Migration and Integration - Oracle Migration and Integration;
        \end{itemize}

        \item \textbf{VI Settimana - Operatività 1 (24 ore)} 
        \begin{itemize}
            \item Oracle Cloud Database Migration and Integration - RMAN, Data Pump and Cloning to Migrate to OCI;
            \item Oracle Cloud Database Migration and Integration - Migrating to OCI using SQL Developer, Cloud Backup, and GoldenGate;    
            \item Oracle Cloud Database Migration and Integration - Understanding Source and Target Database and Migrate with Oracle ZDM;
        \end{itemize}

        \item \textbf{VII Settimana (24 ore)} 
        \begin{itemize}
            \item ;
        \end{itemize}

        \item \textbf{VIII Settimana (24 ore)} 
        \begin{itemize}
            \item ;
        \end{itemize}

        \item \textbf{IX Settimana (24 ore)} 
        \begin{itemize}
            \item ;
        \end{itemize}

        \item \textbf{X Settimana (24 ore)} 
        \begin{itemize}
            \item ;
        \end{itemize}

        \item \textbf{XI Settimana (24 ore)} 
        \begin{itemize}
            \item ;
        \end{itemize}

        \item \textbf{XII Settimana (24 ore)} 
        \begin{itemize}
            \item ;
        \end{itemize}

        \item \textbf{XIII Settimana (24 ore)} 
        \begin{itemize}
            \item ;
        \end{itemize}
    \end{itemize}
}

% Indicare il totale complessivo (deve essere compreso tra le 300 e le 320 ore)
\newcommand{\totaleOre}{300}

\newcommand{\obiettiviObbligatori}{
	 \item \underline{\textit{O01}}: Apprendere le nozioni fondamentali per la gestione di un database Oracle;
	 \item \underline{\textit{O02}}: Conoscere le soluzione per i database Oracle su OCI (Oracle Cloud Infrastructure) e saper effettuare provisioning di database;
	 \item \underline{\textit{O03}}: Conoscere le principali tecniche di migrazione di database su cloud;
	 \item \underline{\textit{O04}}: Realizzare un caso concreto di migrazione di un database su cloud Oracle;	 
}

\newcommand{\obiettiviDesiderabili}{
	 \item \underline{\textit{D01}}: Conoscere e saper gestire Oracle Autonomous DB;
}

\newcommand{\obiettiviFacoltativi}{
	 \item \underline{\textit{F01}}: Realizzare la migrazione di un database su Oracle Autonomous DB;
}